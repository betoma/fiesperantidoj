\subsection{\phipa{\labrox} becomes \bripa{w} when it would naturally fit in the sonority hierarchy of a consonant cluster and \bripa{v} elsewhere}

\begin{center}
    \begin{tabular}{l}
        \labrox{} $\to$ w / C\_\\
        \labrox{} $\to$ v / \emph{elsewhere}    
    \end{tabular}
\end{center}
e.g., \espq{kvarto} \bripa{\prstr k\labrox ar.to} $\to$ \bripa{\prstr kwar.to}, \espq{evakui} \bripa{\scstr e.\labrox a\prstr ku.i} $\to$ \bripa{\scstr e.va\prstr ku.i}

\subsection{Stress moves to the last syllable of correlatives and forms of \espq{esti} other than the infinitive and imperative}

e.g., \espq{tiel} \phipa{\prstr ti.el} $\to$ \phipa{ti\prstr el}, \espq{estas} \phipa{\prstr e.stas} $\to$ \phipa{e\prstr stas}, etc.

\subsection{\phipa{i} and \phipa{u} $\to$ \phipa{j} and \phipa{w} before a stressed vowel}

e.g., \espq{duono} \bripa{du\prstr o.no} $\to$ \bripa{\prstr dwo.no}, \espq{tiel} \bripa{ti\prstr el} $\to$ \bripa{tjel}, etc.

\subsection{Nasals assimilate in place of articulation to a following obstruent}
\begin{center}
    \begin{tabular}{l}
        N $\to$ m / \_\{p, b, f\}\\[0.1cm]
        N $\to$ n / \_\{t, d, t\tiebar\esh, d\tiebar\ezh, s, z, \esh, \ezh\}\\[0.1cm]
        N $\to$ \engma / \_\{k, g\}
    \end{tabular}
\end{center}
e.g., \espq{enblovi} \bripa{en\prstr blowi} $\to$ \bripa{em\prstr blo.wi}

\subsection{Obstruents assimilate in voicing to a following obstruent}

e.g., \espq{absolute} \bripa{ab.so\prstr lu.te} $\to$ \bripa{ap.so\prstr lu.te}

\subsection{Sonorants assimilate in voicing to a preceding stop}

e.g., \espq{plua} \bripa{\prstr plu.a} $\to$ \bripa{\prstr pl\vless u.a}

\subsection{Clusters of mixed sibilants assimilate to the last sibilant}

e.g., \espq{dis\^{j}eti} \bripa{dis\prstr\ezh e.ti} $\to$ \bripa{di\ezh\prstr\ezh e.ti}

\subsection{Vowel hiatuses are broken up by epenthetic consonants}

\begin{itemize}
    \item If both consonants are the same, the epenthetic consonant is a glottal stop.
    \begin{center}
    \begin{tabular}{lllll}
        \bripa{a.a} $\to$ \bripa{a\glotstop a} &
        \bripa{e.e} $\to$ \bripa{e\glotstop e} &
        \bripa{i.i} $\to$ \bripa{i\glotstop i} &
        \bripa{o.o} $\to$ \bripa{o\glotstop o} &
        \bripa{u.u} $\to$ \bripa{u\glotstop u}
    \end{tabular}
    \end{center}
    \item If the first non-low vowel is front, the epenthetic consonant is \bripa{j}
    \begin{center}
    \begin{tabular}{llll}
        \bripa{i.e} $\to$ \bripa{ije} &
        \bripa{i.a} $\to$ \bripa{ija} &
        \bripa{i.o} $\to$ \bripa{ijo} &
        \bripa{i.u} $\to$ \bripa{iju}\\
        \bripa{e.i} $\to$ \bripa{eji} &
        \bripa{e.a} $\to$ \bripa{eja} &
        \bripa{e.o} $\to$ \bripa{ejo} &
        \bripa{e.u} $\to$ \bripa{eju}\\
        \bripa{a.i} $\to$ \bripa{aji} &
        \bripa{a.e} $\to$ \bripa{aje}
    \end{tabular}
    \end{center}
    \item If the first non-low vowel is back, the epenthetic consonant is \bripa{w}
    \begin{center}
    \begin{tabular}{llll}
        \bripa{o.i} $\to$ \bripa{owi} &
        \bripa{o.e} $\to$ \bripa{owe} &
        \bripa{o.a} $\to$ \bripa{owa} &
        \bripa{o.u} $\to$ \bripa{owu}\\
        \bripa{u.i} $\to$ \bripa{uwi} &
        \bripa{u.e} $\to$ \bripa{uwe} &
        \bripa{u.a} $\to$ \bripa{uwa} &
        \bripa{u.o} $\to$ \bripa{uwo}\\
        \bripa{a.o} $\to$ \bripa{awo} &
        \bripa{a.u} $\to$ \bripa{awu}
    \end{tabular}
    \end{center}
\end{itemize}

\subsection{Velar obstruents become palatal before front vowels}

\begin{center}
k g $\to$ c \paljstop{} / \_\{i,e\}
\end{center}

e.g., \espq{kilogramo} \bripa{\scstr ki.lo\prstr gra.mo} $\to$ \bripa{\scstr ci.lo\prstr gra.mo}

\subsection{Diphthongs turn into stressed (long) monophthongs}

\begin{center}
\begin{tabular}{lllll}
    ij $\to$ i\lgth &
    uj $\to$ y\lgth &
    aj ej $\to$ e\lgth &
    oj ew $\to$ ø\lgth &
    aw $\to$ o\lgth \\
    \multicolumn{5}{c}{/ \_\{C,\#\}}
\end{tabular}%
\end{center}

Note that this only occurs when the glide occurs pre-consonantally or word-finally, \emph{not} when it occurs intervocalically (in which case it's analyzed as a consonant rather than as part of a diphthong).\\
e.g, \espq{kajto} \bripa{\prstr kaj.to} $\to$ \bripa{\prstr ke\lgth.to} but \espq{kajako} \bripa{ka\prstr ja.ko} $\to$ \bripa{ka\prstr ja.ko}

\subsection{Glides fricate intervocalically or word-initially}

\begin{center}
j w $\to$ \paljfric{} v / \#\_ , V\_V
\end{center}

e.g., \espq{ejakuli} \bripa{\scstr e.ja\prstr ku.li} $\to$ \bripa{\scstr e.\paljfric a\prstr ku.li} but \espq{ajna} \bripa{\prstr aj.na} $\to$ \bripa{\prstr e\lgth na} but \espq{anta\u{u}a} \bripa{an\prstr ta.wa} $\to$ \bripa{an\prstr ta.va}

###---pick up here---###

\subsection{Glottal stop and glottal fricative merge to glottal fricative}

\begin{center}
\phipa{\glotstop} $\to$ \phipa{h}
\end{center}

e.g., \espq{heroo} \bripa{he\prstr\alvrap o.\glotstop o} $\to$ \bripa{he\prstr\alvrap o.ho} in Lowlands, \bripa{\glotstop e\prstr\alvrap o.\glotstop o} in Highlands

\subsection{{\sc Lowlands:} Standalone obstruents are voiced intervocalically} 

\begin{center}
    p t c k t\tiebar s t\tiebar\esh{} f s \esh{} h $\to$ b d \paljstop{} g d\tiebar z d\tiebar\ezh{} v z \ezh{} \voih{} / V\_V
\end{center}

e.g., \espq{\^{s}ipo} \bripa{\prstr\esh i.po} $\to$ \bripa{\prstr\esh i.bo} 

\subsection{Intervocalic geminates become single occurrences of the consonant in question}

\begin{center}
    C$^1$C$^2$ $\to$ C$^1$ / C$^1$ = C$^2$
\end{center}

e.g., \espq{dis\^{s}uti} \bripa{di\esh\prstr\esh u.di} $\to$ \bripa{di\prstr\esh u.di}

\subsection{{\sc Lowlands:} Non-glottal voiceless fricatives are affricativized after nasal consonants, non-glottal voiced fricatives are nasalized}

\begin{center}
    f s \esh{} $\to$ p\tiebar f t\tiebar s t\tiebar\esh{} / N\_ \\[0.1cm]
    v z \ezh{} $\to$ \~{v} \~{z} \~{\ezh} / N\_
\end{center}

e.g., \espq{komforti} \bripa{kom\prstr fo\alvrap.ti} $\to$ \bripa{kom\prstr p\tiebar fo\alvrap.ti}, \espq{bronza} \bripa{b\alvrap\vless on.za} $\to$ \bripa{b\alvrap\vless on.d\tiebar za}

\subsection{Ablaut}

If the last vowel in a word is back rounded, the preceding vowel is rounded (regardless of intervening consonants).\\
e.g., \espq{kato} \bripa{\prstr ka.to} $\to$ \bripa{\prstr k\ahoh.to}, \espq{iros} \bripa{\prstr i.ros} $\to$ \bripa{\prstr y.ros}, \espq{ekzemplo} \bripa{ek\prstr sem.plo} $\to$ \bripa{ek\prstr søm.plo}

If the last vowel in a word is front, the preceding vowel is fronted (regardless of intervening consonants).\\
e.g., \espq{havis} \bripa{\prstr ha.vis} $\to$ \bripa{\prstr h\aesh.vis}, \espq{ofte} \bripa{\prstr of.te} $\to$ \bripa{\prstr øf.te}, \espq{seksumi} \bripa{sek\prstr su.mi} $\to$ \bripa{sek\prstr sy.mi}

\subsection{Vowels are nasalized before nasal consonants, and nasal consonants are deleted when they precede obstruents}

\begin{center}
    \begin{tabular}{l}
        VN $\to$ \~{V} / \_P \\
        VN $\to$ \~{V}N / \emph{elsewhere}
    \end{tabular}
\end{center}

e.g., \espq{anka\u{u}} \bripa{\prstr\ahoh n.ko\lgth} $\to$ \bripa{\prstr\~{\ahoh}.ko\lgth}

\subsection{{\sc Highlands:} Standalone obstruents are voiced intervocalically}

\begin{center}
    p t c k t\tiebar s t\tiebar\esh{} f s \esh{} $\to$ b d \paljstop{} g d\tiebar z d\tiebar\ezh{} v z \ezh{} / V\_V
\end{center}

e.g., \espq{\^{s}ipo} \bripa{\prstr\esh y.po} $\to$ \bripa{\prstr\esh y.bo} (identical in Lowlands), \espq{dis\^{s}uti} \bripa{di\prstr\esh y.ti} $\to$ \bripa{di\prstr\ezh y.di} (cf. Lowlands \bripa{di\prstr\esh y.di}), \espq{anka\u{u}} \bripa{\prstr\~{\ahoh}.ko\lgth} $\to$ \bripa{\prstr\~{\ahoh}.go\lgth} (cf. Lowlands \bripa{\prstr\~{\ahoh}.ko\lgth})

\subsection{Unstressed short vowels become schwa}

e.g., \espq{Esperanto} \bripa{\scstr e.spe\prstr r\~{\ahoh}.to} $\to$ \bripa{\scstr e.sp\schwa\prstr r\~{\ahoh}.t\schwa}, \espq{tajfuno} \bripa{te\lgth\prstr fu.no} $\to$ \bripa{te\lgth\prstr fu.n\schwa}

\subsection{{\sc Highlands:} Word-initial syllables consisting of an obstruent or nasal, a schwa, and then a continuant preceding a consonant undergo metathesis of the schwa and continuant}

\begin{center}
    C$_1$\schwa C$_2$ $\to$ C$_1$C$_2$\schwa{} / \#\_C\\
    C$_1$ = obstruent or nasal, C$_2$ = continuant
\end{center}

\subsection{{\sc Highlands:} All glottal fricatives become glottal stops}

\begin{center}
    \phipa{h} $\to$ \phipa{\glotstop}
\end{center}