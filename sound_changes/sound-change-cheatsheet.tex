\documentclass[a4paper,11pt,article,oneside]{memoir}
\usepackage{fontspec}
\usepackage{amsmath} % a pretty standard package to enlarge your inventory of symbols
\usepackage{amssymb} % another common package for symbols
\usepackage[hidelinks]{hyperref} % enables hyperlinks in your document (no worries -- they show up only on the screen. When you print a hard copy, the colored boxes aren't there)
\usepackage{url} % helps typeset URLs properly, typically with the command \url
\usepackage[margin=.5in]{geometry} % page layout
\usepackage{booktabs} % creates beautiful and professional tables
\usepackage{multirow}
\usepackage{multicol}
\usepackage{textcomp}
\usepackage{expex}
\usepackage{threeparttable}
\usepackage[calc,english]{datetime2}
\usepackage{bookmark}

\setmainfont{Charis SIL}[CharacterVariant=43:1]

\setlength{\parindent}{0em}
\setlength{\parskip}{1ex}
%\linespread{1.1}

\input{../preamble_misc.tex}

\title{\langname{} Sound Changes:\\{\Large A Cheat Sheet}}
\author{Bethany E. Toma}
\date{\today}

\counterwithout{section}{chapter}
\maxsecnumdepth{section}

\begin{document}

\maketitle

While \langname{} is distinct enough from its Esperanto ancestor to be reasonably considered a separate language at this juncture, its orthography does not reflect this. While syntactic and semantic changes are reflected to a fairly reasonable extent in the orthography, phonological changes are not accounted for. As a reult, to reasonably approximate the pronunciation of a \langname{} word from how it is written, one must understand which sound changes have occurred from the `original' (hereafter `Zamenhofian') Esperanto. 

Note that some sound changes listed apply only to the Lowlands dialect but not to the Highlands dialect, or vice-versa. This is indicated when it is the case.

\section*{Zamenhofian Esperanto Orthography}

The Esperanto pronunciation of a word tends to be roughly equivalent to its spelling in IPA, with the following exceptions:
\begin{center}
\begin{threeparttable}
\begin{tabular}{ccccccccc}
    \ortho{c} & \ortho{\^{c}} & \ortho{\^{g}} & \ortho{\^{h}}\tnote{1} & \ortho{\^{j}} & \ortho{\^{s}} & \ortho{r} & \ortho{\u{u}} & \ortho{v}\\[0.2cm]
    \phipa{t\tiebar s} & \phipa{t\tiebar\esh} & \phipa{d\tiebar\ezh} & \phipa{x} & \phipa{\ezh} & \phipa{\esh} & \phipa{\alvrap} & \phipa{w} & \phipa{\labrox}
\end{tabular}
\begin{tablenotes}
    \item[1] This phoneme is only retained in a very few words at this point (e.g., \espq{eĥo}, \espq{ĥaoso}) and is elsewhere generally replaced with variants with k or h.
\end{tablenotes}
\end{threeparttable}
\end{center}

Primary stress falls on the penultimate syllable, and secondary stress \emph{tends} to fall on alternating syllables preceding it.

\section{Sound Changes}

\subsection{0--100 AMC}


\subsubsection{\phipa{\labrox} becomes \bripa{w} when it would fit in the sonority hierarchy of a consonant cluster and \bripa{v} elsewhere}

\begin{center}
    \begin{tabular}{l}
        \labrox{} $\to$ w / C\_V\\
        \labrox{} $\to$ v / \emph{elsewhere}    
    \end{tabular}
\end{center}
e.g., \espq{kvarto} \bripa{\prstr k\labrox ar.to} $\to$ \bripa{\prstr kwar.to}, \espq{evakui} \bripa{\scstr e.\labrox a\prstr ku.i} $\to$ \bripa{\scstr e.va\prstr ku.i}

\subsubsection{Stress moves to the last syllable of correlatives and finite forms of \espq{esti}}

e.g., \espq{tiel} \phipa{\prstr ti.el} $\to$ \phipa{ti\prstr el}, \espq{estas} \phipa{\prstr e.stas} $\to$ \phipa{e\prstr stas}, etc., but \espq{estu} remains \phipa{\prstr e.stu}

\subsubsection{\phipa{i} and \phipa{u} $\to$ \phipa{j} and \phipa{w} before a stressed vowel}

e.g., \espq{duono} \bripa{du\prstr o.no} $\to$ \bripa{\prstr dwo.no}, \espq{tiel} \bripa{ti\prstr el} $\to$ \bripa{tjel}, etc.

\subsubsection{Nasals assimilate in place of articulation to a following obstruent}

\begin{center}
    \begin{tabular}{l}
        N $\to$ m / \_\{p, b, f\}\\[0.1cm]
        N $\to$ n / \_\{t, d, t\tiebar\esh, d\tiebar\ezh, s, z, \esh, \ezh\}\\[0.1cm]
        N $\to$ \engma / \_\{k, g\}
    \end{tabular}
\end{center}
e.g., \espq{enblovi} \bripa{en\prstr blowi} $\to$ \bripa{em\prstr blo.wi}

\subsubsection{Obstruents assimilate in voicing to a following obstruent}

e.g., \espq{absolute} \bripa{ab.so\prstr lu.te} $\to$ \bripa{ap.so\prstr lu.te}

\subsection{Clusters of mixed sibilants assimilate to the last sibilant}

e.g., \espq{dis\^{j}eti} \bripa{dis\prstr\ezh e.ti} $\to$ \bripa{di\ezh\prstr\ezh e.ti}

\subsection{Vowel hiatuses are broken up by epenthetic consonants}

\begin{itemize}
    \item If both consonants are the same, the epenthetic consonant is a glottal stop.
    \begin{center}
    \begin{tabular}{lllll}
        \bripa{a.a} $\to$ \bripa{a\glotstop a} &
        \bripa{e.e} $\to$ \bripa{e\glotstop e} &
        \bripa{i.i} $\to$ \bripa{i\glotstop i} &
        \bripa{o.o} $\to$ \bripa{o\glotstop o} &
        \bripa{u.u} $\to$ \bripa{u\glotstop u}
    \end{tabular}
    \end{center}
    \item If the first non-low vowel is front, the epenthetic consonant is \bripa{j}
    \begin{center}
    \begin{tabular}{llll}
        \bripa{i.e} $\to$ \bripa{ije} &
        \bripa{i.a} $\to$ \bripa{ija} &
        \bripa{i.o} $\to$ \bripa{ijo} &
        \bripa{i.u} $\to$ \bripa{iju}\\
        \bripa{e.i} $\to$ \bripa{eji} &
        \bripa{e.a} $\to$ \bripa{eja} &
        \bripa{e.o} $\to$ \bripa{ejo} &
        \bripa{e.u} $\to$ \bripa{eju}\\
        \bripa{a.i} $\to$ \bripa{aji} &
        \bripa{a.e} $\to$ \bripa{aje}
    \end{tabular}
    \end{center}
    \item If the first non-low vowel is back, the epenthetic consonant is \bripa{w}
    \begin{center}
    \begin{tabular}{llll}
        \bripa{o.i} $\to$ \bripa{owi} &
        \bripa{o.e} $\to$ \bripa{owe} &
        \bripa{o.a} $\to$ \bripa{owa} &
        \bripa{o.u} $\to$ \bripa{owu}\\
        \bripa{u.i} $\to$ \bripa{uwi} &
        \bripa{u.e} $\to$ \bripa{uwe} &
        \bripa{u.a} $\to$ \bripa{uwa} &
        \bripa{u.o} $\to$ \bripa{uwo}\\
        \bripa{a.o} $\to$ \bripa{awo} &
        \bripa{a.u} $\to$ \bripa{awu}
    \end{tabular}
    \end{center}
\end{itemize}

\subsubsection{Velar obstruents become palatal before front vowels/glides}

\begin{center}
k g $\to$ c \paljstop{} / \_\{i,e,j\}
\end{center}

e.g., \espq{kilogramo} \bripa{\scstr ki.lo\prstr gra.mo} $\to$ \bripa{\scstr ci.lo\prstr gra.mo}

\subsubsection{Vowels shift closer to glides when they precede them}

Back vowels are fronted before \phipa{j}, and front vowels are rounded before \phipa{w}. Low vowels are raised to mid prior to either glide and are fronted before \phipa{j} and backed and rounded before \phipa{w}.

\begin{center}
    u $\to$ y / \_\,j\\
    o $\to$ ø / \_\,j\\
    a $\to$ e / \_\,j\\
    a $\to$ o / \_w\\
    e $\to$ ø / \_w
\end{center}

e.g., \espq{kajto} \bripa{\prstr kaj.to} $\to$ \bripa{\prstr kej.to}, \espq{abelujo} \bripa{\scstr a.be\prstr lu.jo} $\to$ \bripa{\scstr a.be\prstr ly.jo}, \espq{vojo} \bripa{\prstr vo.jo} $\to$ \bripa{\prstr vø.jo}, \espq{ambaŭ} \bripa{\prstr am.baw} $\to$ \bripa{\prstr am.bow}, \espq{eŭropa} \bripa{ew\prstr ro.pa} $\to$ \bripa{øw\prstr ro.pa}

\subsection{100--200 AMC}

\subsection{Intensification suffixes}

In Zamenhofian Esperanto, the augmentative \espq{-eg-} and diminutive \espq{-et-} are used as degree modifiers for adjectives, with \espq{-eg-} serving as an intensifier and \espq{-et-} as a downtowner regardless of the semantics of the adjective in question. 

Here, however, this usage begins to change, with adjectives being intensified using affixes that are semantically suited to their meaning. This is most apparently different with \espq{malgranda}, meaning \engq{small}. In Zamenhofian Esperanto, \espq{malgrandeta} means \engq{a little small} while \espq{malgrandega} means \engq{tiny}, whereas now \espq{malgrandega} has fallen out of usage and \espq{malgrandeta} has taken over its meaning of \engq{tiny}. The thought process behind this is that the diminutive of \engq{small} should be \emph{smaller} than the bare form of small, not larger, and thus it can serve as an intensifier here.

In addition to this example with \espq{malgranda}, other affixes like \espq{-el-} and \espq{-aĉ-} are used with adjectives that are semantically suited to them (e.g., \espq{belela} has replaced \espq{belega} and \espq{naŭzaĉa} has replaced \espq{naŭzega}).

\subsection{No more adjective agreement}

Pretty much what it says on the tin: adjectives no longer agree with the nouns they modify in number and case. 

\ex
\espq{La bela virinoj manĝis la dolĉa kukojn.}\\
\engq{The beautiful women ate the sweet cakes.}
\xe

\subsection{Emphatic negation}

In addition to ordinary negation with \espq{ne}, negation begins to be frequently emphasized by inclusion of \espq{neniel} as well. This roughly corresponds to inclusion of the English negative polarity item \engq{at all}.

\ex
\espq{Mi ne volas manĝi tion neniel.}\\
\engq{I don't want to eat that at all.}
\xe

\subsection{Use of \espq{si} logophorically}

In addition to its use as a simple reflexive, \espq{si} is expanded to be used as a logophoric pronoun, also being used in subclauses to indicate that the referent has not changed.

\pex
\a
\espq{Johano diris, ke si foriru.}\\
\engq{John$_i$ said that he$_i$ has to leave.}
\a
\espq{Johano diris, ke li foriru.}\\
\engq{John$_i$ said that he$_j$ has to leave.}
\xe

\subsection{200--300 AMC}

\subsubsection{{\sc Lowlands:} Nasal consonants are deleted when they precede obstruents}

\subsubsection{Ablaut}

If the last vowel in a word is rounded, the preceding vowel is rounded (regardless of intervening consonants).\\
e.g., \espq{kato} \bripa{\prstr ka.to} $\to$ \bripa{\prstr k\ahoh.to}, \espq{iros} \bripa{\prstr i.ros} $\to$ \bripa{\prstr y.ros}, \espq{ekzemplo} \bripa{ek\prstr sẽm.plo} $\to$ \bripa{ek\prstr sø̃m.plo}

If the last vowel in a word is front, the preceding vowel is fronted (regardless of intervening consonants).\\
e.g., \espq{havis} \bripa{\prstr ha.vis} $\to$ \bripa{\prstr h\aesh.vis}, \espq{ofte} \bripa{\prstr of.te} $\to$ \bripa{\prstr øf.te}, \espq{seksumi} \bripa{sek\prstr sũ.mi} $\to$ \bripa{sek\prstr sỹ.mi}

\begin{center}
    V$^{-front}$ $\to$ V$^{+front}$ / \_(C(C))\$(C(C(C)))V$^{+front}$\#\\
    V$^{-round}$ $\to$ V$^{+round}$ / \_(C(C))\$(C(C(C)))V$^{+round}$\#
\end{center}

\begin{center}
    \phipa{i}$^{+round} = $ \bripa{y}\\
    \phipa{e}$^{+round} = $ \bripa{ø}\\
    \phipa{a}$^{+round} = $ \bripa{\ahoh}\\
    \phipa{a}$^{+front} = $ \bripa{\aesh}\\
    \phipa{o}$^{+front} = $ \bripa{ø}\\
    \phipa{u}$^{+front} = $ \bripa{y}\\
\end{center}

\subsubsection{{\sc Highlands:} Word-initial syllables consisting of an obstruent or nasal, a vowel, and then a continuant in the syllable coda undergo metathesis of the schwa and continuant}

\begin{center}
    C$_1$VC$_2$ $\to$ C$_1$C$_2$V / \#\_(C)\$\\
    C$_1$ = obstruent or nasal, C$_2$ = continuant
\end{center}

e.g. \espq{forkuri} \bripa{fo\alvrap\prstr ku.\alvrap i} $\to$ \bripa{fro\prstr ku.\alvrap i}


\subsubsection{}

\subsection{300--400 AMC}


\subsubsection{\phipa{\alvrap} disappears in the coda and the preceding vowel, if short, undergoes compensatory lengthening}

\begin{center}
    V\alvrap{} $\to$ V\lgth{} / \_(C)\$
\end{center}

e.g., \espq{forkuri} (Lowlands) \bripa{fo\alvrap\prstr ku.\alvrap i} $\to$ \bripa{fo\lgth\prstr ku.\alvrap i}

\subsubsection{l-vocalization occurs when l occurs in the coda of a syllable}

\begin{center}
    l $\to$ w / V\_(C)\$
\end{center}

e.g., \espq{bulbo} \bripa{\prstr bul.bo} $\to$ \bripa{\prstr buw.bo}

\subsubsection{{\sc Highlands:} Standalone obstruents are voiced intervocalically}

\begin{center}
    p t c k t\tiebar s t\tiebar\esh{} f s \esh{} $\to$ b d \paljstop{} g d\tiebar z d\tiebar\ezh{} v z \ezh{} / V\_V
\end{center}

e.g., \espq{\^{s}ipo} \bripa{\prstr\esh y.po} $\to$ \bripa{\prstr\esh y.bo} (identical in Lowlands), \espq{dis\^{s}uti} \bripa{di\prstr\esh y.ti} $\to$ \bripa{di\prstr\ezh y.di} (cf. Lowlands \bripa{di\prstr\esh u.di}), \espq{anka\u{u}} \bripa{\prstr\~{\ahoh}.ko\lgth} $\to$ \bripa{\prstr\~{\ahoh}.go\lgth} (cf. Lowlands \bripa{\prstr\~{\ahoh}.ko\lgth})

\subsubsection{Unstressed short vowels become schwa}

e.g., \espq{Esperanto} \bripa{\scstr e.spe\prstr r\~{\ahoh}.to} $\to$ \bripa{\scstr e.sp\schwa\prstr r\~{\ahoh}.t\schwa}, \espq{tajfuno} \bripa{te\lgth\prstr fu.no} $\to$ \bripa{te\lgth\prstr fu.n\schwa}

\subsubsection{{\sc Lowlands:} After a nasal vowel, voiced stops become nasal stops}

\begin{center}
    b d \paljstop{} g $\to$ m n \egna{} \engma{} / \~{V}\_
\end{center}

e.g., \espq{embrio} \bripa{\~{\schwa}\prstr b\alvrap y.\paljfric\schwa} $\to$ \bripa{\~{\schwa}\prstr m\alvrap y\paljfric\schwa}

\subsection{400--500 AMC}

\input{sound_change_stages/amc_400.tex}

\subsection{500--600 AMC}

\input{sound_change_stages/amc_500.tex}

\subsection{600--700 AMC}

\input{sound_change_stages/amc_600.tex}

\subsection{700--800 AMC}

\input{sound_change_stages/amc_700.tex}

\subsection{800--900 AMC}

\input{sound_change_stages/amc_800.tex}

\subsection{900--1000 AMC}

\input{sound_change_stages/amc_900.tex}

\section{Sparky's Transcription}

I'll probably make an in-world orthography later, but regardless of whether I do, I need a roughly phonetic or phonemic transcription which I can use to not have to constantly write in IPA. That'll go here once I actually make it lol.

\clearpage
\section{Pronunciation Example}

The following is the Zamenhofian Esperanto text of the Lord's Prayer. Given that the text is highly ritualized, it has not been affected by the grammatical and lexical changes to the language, but it remains pronounced differently in the different dialects due to the sound changes that have occurred.

\subsection{Zamenhofican Esperanto c. 0 AMC}

\begin{multicols*}{2}
    \textit{Patro nia, kiu estas en la ĉielo, \\
    sanktigata estu Via Nomo. \\
    Venu Via regno. \\
    Fariĝu Via volo, \\
    kiel en la ĉielo, tiel ankaŭ sur la tero.}

    \textit{Nian panon ĉiutagan donu al ni hodiaŭ. \\
    Kaj pardonu al ni niajn ŝuldojn, \\
    kiel ankaŭ ni pardonas al niaj ŝuldantoj. \\
    Kaj ne konduku nin en tenton, \\
    sed liberigu nin de la malbono.}

    \textit{Ĉar Via estas la regno 
    kaj la potenco \\
    kaj la gloro eterne.}

    \textit{Amen.}
    
    \columnbreak

    \prstr pa.t\alvrap o \prstr ni.a\\
    \prstr ki.u \prstr e.stas en la t\tiebar\esh i\prstr e.lo\\
    \scstr sank.ti\prstr ga.ta \prstr e.stu \prstr \labrox i.a \prstr no.mo\\
    \prstr\labrox e.nu \prstr\labrox i.a \prstr\alvrap e.gno\\
    fa\prstr\alvrap i.d\tiebar\ezh u \prstr\labrox i.a \prstr\labrox o.lo\\
    \prstr ki.el en la t\tiebar\esh i\prstr e.lo\\
    \prstr ti.el \prstr an.kaw sur la \prstr te.\alvrap o

    \prstr ni.an \prstr pa.non \scstr t\tiebar\esh i.u\prstr ta.gan\\
    \prstr do.nu al ni ho\prstr di.aw\\
    kaj pa\alvrap\prstr don.u al ni \prstr ni.ajn \prstr\esh ul.dojn\\
    \prstr ki.el \prstr an.kaw ni pa\alvrap\prstr do.nas\\
    al \prstr ni.aj \esh ul\prstr dan.toj\\
    kaj ne kon\prstr du.ku nin en \prstr ten.ton\\
    sed \scstr li.be\prstr\alvrap i.gu nin de la mal\prstr bo.no

    t\tiebar\esh a\alvrap{} \prstr\labrox i.a \prstr e.stas la \prstr \alvrap e.gno\\
    kaj la po\prstr ten.t\tiebar so\\
    kaj la \prstr glo.\alvrap o e\prstr te\alvrap.ne

    \prstr a.men

    \end{multicols*}

\subsection{c. 100 AMC}

\prstr pa.t\alvrap o \prstr ni.ja\\
\prstr cju e\prstr stas en la \prstr t\tiebar\esh je.lo\\
\scstr sa\engma k.ti\prstr ga.ta \prstr e.stu \prstr vi.ja \prstr no.mo\\
\prstr ve.nu \prstr vi.ja \prstr\alvrap e.gno\\
fa\prstr\alvrap i.d\tiebar\ezh u \prstr vi.ja \prstr vo.lo\\
cjel en la \prstr t\tiebar\esh je.lo\\
tjel \prstr a\engma.kow sur la \prstr te.\alvrap o

\prstr ni.jan \prstr pa.non \scstr t\tiebar\esh ju\prstr ta.gan\\
\prstr do.nu al ni ho\prstr di.jow\\
kej pa\alvrap\prstr don.u al ni \prstr ni.jejn \prstr\esh ul.døjn\\
cjel \prstr a\engma.kow ni pa\alvrap\prstr do.nas\\
al \prstr ni.jej \esh ul\prstr dan.tøj\\
kej ne kon\prstr du.ku nin en \prstr ten.ton\\
sed \scstr li.be\prstr\alvrap i.gu nin de la mal\prstr bo.no

t\tiebar\esh a\alvrap{} \prstr vi.ja e\prstr stas la \prstr \alvrap e.gno\\
kej la po\prstr ten.t\tiebar so\\
kej la \prstr glo.\alvrap o e\prstr te\alvrap.ne

\prstr a.men


\subsection{c. 200 AMC}

\begin{multicols*}{2}

{\sc Lowlands}

\prstr pa.t\alvrap o \prstr ni.\paljfric a\\
\prstr cju e\prstr stas ẽn la \prstr t\tiebar\esh je.lo\\
\scstr sã\engma k.ti\prstr ga.da \prstr e.stu \prstr vi.\paljfric a \prstr no.mo\\
\prstr vẽ.nu \prstr vi.\paljfric a \prstr\alvrap e.gno\\
fa\prstr\alvrap i.d\tiebar\ezh u \prstr vi.\paljfric a \prstr vo.lo\\
cjel ẽn la \prstr t\tiebar\esh je.lo\\
tjel \prstr ã\engma.ko\lgth{} sur la \prstr te.\alvrap o

\prstr ni.\paljfric ãn \prstr pã.nõn \scstr t\tiebar\esh ju\prstr ta.gãn\\
\prstr dõ.nu al ni ho\prstr di.\paljfric o\lgth{}\\
ke\lgth{} pa\alvrap\prstr dõn.u al ni \prstr ni.\paljfric ẽ\lgth{}n \prstr\esh ul.dø̃\lgth{}n\\
cjel \prstr ã\engma.ko\lgth{} ni pa\alvrap\prstr dõ.nas\\
al \prstr ni.\paljfric e\lgth{} \esh ul\prstr dãn.tø\lgth{}\\
ke\lgth{} ne kõn\prstr du.gu nĩn en \prstr tẽn.tõn\\
sed \scstr li.be\prstr\alvrap i.gu nĩn de la mal\prstr bõ.no

t\tiebar\esh a\alvrap{} \prstr vi.\paljfric a e\prstr stas la \prstr \alvrap e.gno\\
ke\lgth{} la po\prstr dẽn.d\tiebar zo\\
ke\lgth{} la \prstr glo.\alvrap o e\prstr te\alvrap.ne

\prstr ã.mẽn

\columnbreak

{\sc Highlands}

\prstr pa.t\alvrap o \prstr ni.\paljfric a\\
\prstr cju e\prstr stas ẽn la \prstr t\tiebar\esh je.lo\\
\scstr sã\engma k.ti\prstr ga.ta \prstr e.stu \prstr vi.\paljfric a \prstr no.mo\\
\prstr vẽ.nu \prstr vi.\paljfric a \prstr\alvrap e.gno\\
fa\prstr\alvrap i.d\tiebar\ezh u \prstr vi.\paljfric a \prstr vo.lo\\
cjel ẽn la \prstr t\tiebar\esh je.lo\\
tjel \prstr ã\engma.ko\lgth{} sur la \prstr te.\alvrap o

\prstr ni.\paljfric ãn \prstr pã.nõn \scstr t\tiebar\esh ju\prstr ta.gãn\\
\prstr dõ.nu al ni ho\prstr di.\paljfric o\lgth{}\\
ke\lgth{} pa\alvrap\prstr dõn.u al ni \prstr ni.\paljfric ẽ\lgth{}n \prstr\esh ul.dø̃\lgth{}n\\
cjel \prstr ã\engma.ko\lgth{} ni pa\alvrap\prstr dõ.nas\\
al \prstr ni.\paljfric e\lgth{} \esh ul\prstr dãn.tø\lgth{}\\
ke\lgth{} ne kõn\prstr du.ku nĩn ẽn \prstr tẽn.tõn\\
sed \scstr li.be\prstr\alvrap i.gu nĩn de la mal\prstr bõ.no

t\tiebar\esh a\alvrap{} \prstr vi.\paljfric a e\prstr stas la \prstr \alvrap e.gno\\
ke\lgth{} la po\prstr tẽn.t\tiebar so\\
ke\lgth{} la \prstr glo.\alvrap o e\prstr te\alvrap.ne

\prstr ã.mẽn

\end{multicols*}

\subsection{c. 300 AMC}

\subsection{c. 400 AMC}

\subsection{c. 500 AMC}

\subsection{c. 600 AMC}

\subsection{c. 700 AMC}

\subsection{c. 800 AMC}

\subsection{c. 900 AMC}

\subsection{c. 1000 AMC}

\end{document}