\documentclass[a4paper,11pt,article,oneside]{memoir}
\usepackage{fontspec}
\usepackage{amsmath} % a pretty standard package to enlarge your inventory of symbols
\usepackage{amssymb} % another common package for symbols
\usepackage[hidelinks]{hyperref} % enables hyperlinks in your document (no worries -- they show up only on the screen. When you print a hard copy, the colored boxes aren't there)
\usepackage{url} % helps typeset URLs properly, typically with the command \url
\usepackage{geometry} % page layout
\usepackage{booktabs} % creates beautiful and professional tables
\usepackage{multirow}
\usepackage{multicol}
\usepackage{textcomp}
\usepackage{expex}
%\usepackage{tablefootnote}
\usepackage{threeparttable}
\usepackage[calc,english]{datetime2}
\usepackage{bookmark}
\usepackage{lipsum}
\usepackage[base]{babel}

\input{../utils/eurodate.tex}
\input{../utils/lingstyles.tex}

\lingset{lingstyle=default}

\setmainfont{Charis SIL}[CharacterVariant=43:1]

\setlength{\parindent}{0em}
\setlength{\parskip}{1ex}
%\linespread{1.1}

\newcommand{\planetname}{Placeholder}
\newcommand{\planetadjective}{Placeholderian}
\newcommand{\onepeoplename}{}
\newcommand{\onelangname}{X\~{e}\l\~{o}k}


%---LANG'S ORTHO---%
%\^{} for lil hat
%\v{} for the hat that isn't dumb
%\u{u} for that one letter u
%\~{} for tilde
%\l for bar l
\newcommand{\Engma}{Ŋ}
\newcommand{\espq}[1]{\textit{`#1'}}
\newcommand{\engq}[1]{``#1''}
\input{../utils/ipa_macros.tex}
\input{../utils/misc_macros.tex}

\title{\langname{} Sound Changes:\\{\Large A Cheat Sheet}}
\author{Bethany E. Toma}
\date{\today}

\counterwithout{section}{chapter}
\maxsecnumdepth{section}

\begin{document}

\maketitle

While \langname{} is distinct enough from its Esperanto ancestor to be reasonably considered a separate language at this juncture, its orthography does not reflect this. While syntactic and semantic changes are reflected to a fairly reasonable extent in the orthography, phonological changes are not accounted for. As a reult, to reasonably approximate the pronunciation of a \langname{} word from how it is written, one must understand which sound changes have occurred from the `original' (hereafter `Zamenhofian') Esperanto. 

Note that some sound changes listed apply only to the Lowlands dialect but not to the Highlands dialect, or vice-versa. This is indicated when it is the case.

\section*{Zamenhofian Esperanto Orthography}

The Esperanto pronunciation of a word tends to be roughly equivalent to its spelling in IPA, with the following exceptions:
\begin{center}
\begin{threeparttable}
\begin{tabular}{ccccccccc}
    \ortho{c} & \ortho{\^{c}} & \ortho{\^{g}} & \ortho{\^{h}}\tnote{1} & \ortho{\^{j}} & \ortho{\^{s}} & \ortho{r} & \ortho{\u{u}} & \ortho{v}\\[0.2cm]
    \phipa{t\tiebar s} & \phipa{t\tiebar\esh} & \phipa{d\tiebar\ezh} & \phipa{x} & \phipa{\ezh} & \phipa{\esh} & \phipa{\alvrap} & \phipa{w} & \phipa{\labrox}
\end{tabular}
\begin{tablenotes}
    \item[1] This phoneme is only retained in a very few words at this point (e.g., \espq{eĥo}, \espq{ĥaoso}) and is elsewhere generally replaced with variants with k or h.
\end{tablenotes}
\end{threeparttable}
\end{center}

Primary stress falls on the penultimate syllable, and secondary stress \emph{tends} to fall on alternating syllables preceding it.

\section{Sound Changes}

\subsection{0--100 AMC}


\subsubsection{Pronoun Usage}

In contexts where it's unclear whether the referent is male or female, using a gendered \espq{li} or \espq{ŝi} is now dispreferred. The current preference is to use \espq{tiu}. However, like singular-they in English, this isn't generally done with explicitly gendered referents, but rather only with those whose gender is unknown or unspecified.

\ex
\espq{Se via studento scias, kiu ĝin faris, tiu devas paroli.}\\
\engq{If your student$_i$ knows who did it, they$_i$ should talk.}
\xe

Note that, where appropriate, \espq{si} and its forms are still used.

\ex 
\espq{Iu preterlasis sian pluvombrelon!}\\
\engq{Someone$_i$ forgot their$_i$ umbrella!}
\xe

\subsubsection{Conditional Participles}

The formerly-unofficial conditional participle \textit{-unta} is now common and unremarkable. They are not generally used predicatively, as using the conditional verb form with other participles serves those purposes fine, but they are widely used attributively and nominalized.

\pex
\a
\espq{La mortunta knabino}\\
\engq{The girl who would/could have died}
\a
\espq{La regunto}\\
\engq{The would-be ruler}
\xe

\subsubsection{Verbalization of predicate adjectives}

Rather than using \textit{esti} as a copula, adjectives are now generally used directly as stative verbs. Nominals continue to use \textit{esti}, however, and \textit{esti} can be included for emphasis for predicate adjectives.

\ex
\espq{La birdo estas blua} $\to$ \espq{La birdo bluas.}\\
\engq{The bird is blue.}
\xe

\subsubsection{Widespread adoption of `far'}

To avoid the ambiguity of the Zamenhofian Esperanto \espq{de}, \espq{far} is adopted as a shortening of \espq{fare de}, to indicate that something was done/made by someone rather than merely owned by or associated with them.

\pex
\a
\espq{La bindaĵo de la libro de Maria ruĝas.}\\
\engq{The cover of Maria's book (a book which Maria owns but didn't necessarily write) is red.}
\a
\espq{La bindaĵo de la libro far Maria ruĝas.}\\
\engq{The cover of Maria's book (a book which Maria wrote) is red.}
\xe

\subsubsection{Widespread adoption of `cit'}

To attribute a quote to someone, the preposition \espq{cit} is used (again, replacing Esperanto \espq{de}).

\ex
\espq{\engq{Rompu, rompu la murojn inter la popoloj!} cit Zamenhof estas inspiranta citaĵo.}\\
\engq{`Break, break the walls between the peoples!' by Zamenhof is an inspiring quotation.}
\xe

\subsubsection{Country names don't end in \textit{-ujo}}

All country names ending in \textit{-ujo} are replaced with alternatives ending in \textit{-io}. 
\ex
\espq{Francujo} $\to$ \espq{Francio}\\
\engq{France}
\xe
Unrelatedly, a few country names change entirely due to reforming their names being a thing in 2010s Esperanto already:
\ex
\espq{Finnlando} $\to$ \espq{Suomio}\\
\engq{Finland}
\xe

\subsubsection{Free variation between presence and absence of linking -o- in compounds}

Linking -o- isn't completely lost, but it's beginning to be less common than it is in ordinary Esperanto. Forms of the same word with and without linking -o- are common and generally occur in free variation.

\ex
\espq{rozokoloro} \til{} \espq{rozkoloro}
\xe

\subsubsection{Identical adjectival and nominal forms dispreferred}

Words where the adjectival and nominal forms are identical and only distinguished by the difference between the final \textit{-o} and \textit{-a} are highly dispreferred---as are affixes being applied with both an \textit{-o} or \textit{-a}. The less `core' part-of-speech for a given root is formed using another affix that suits it semantically. Popular options include \espq{-ulo} `person', \espq{-ano} `member of', \espq{-eco} `quality of, -ness', \espq{-ema} `inclined toward', etc. Sometimes compounding performs the same role---for instance, \espq{roza} \engq{pink (adj.)} vs. \espq{roz(o)koloro} \engq{pink (n.)} vs. \espq{roz(o)floro} \engq{rose}

\subsubsection{Less-recognized/Ido-loaned suffixes more common}

\paragraph{\espq{-el-}}

X\textit{-elo} = \engq{good/beautiful X}: \espq{skribo} \engq{writing}, \espq{skribelo} \engq{calligraphy}

\paragraph{\espq{-oz-}} 

X\textit{-oza} = \engq{full of X}: \espq{monto} \engq{mountain}, \espq{montoza} \engq{mountainous}

\paragraph{\espq{-end-}}

X\textit{-enda} = \engq{needing to be X-ed}: \espq{pagi} \engq{to pay}, \espq{pagenda} \engq{needing paid}

\subsubsection{\espq{ali-} is an official correlative now}

The forms \espq{alio}, \espq{aliu}, \espq{alia}, \espq{aliam}, \espq{aliom} \espq{aliel}, etc. are now officially sanctioned rather than not being official. Because \espq{alie} already exists, the form for `another place' is \espq{aliloke}.

\subsubsection{Pro-drop-ness begins: if \espq{ĝi} would be the subject, don't bother}

The inanimate 3rd person pronoun \espq{ĝi} is now pretty much universally dropped when it would be the subject of a sentence. I'm not actually sure to what extent this is done in vanilla Esperanto tbh---I know it's already a think for zero-valency verbs like weather verbs, but now it's a thing for any sentence where \espq{ĝi} would be the subject.

\ex
\espq{Mi bakis kukon. Bongustis.}\\
\engq{I baked a cake. It was tasty.}
\xe

This is \emph{not} true in instances where it would be blocked by a subject-pivot, though!

\pex
\a
\ljudge{\#}
\espq{Mi bakis kukon kaj bongustis.}\\
\engq{I baked a cake and was tasty.}
\a
\espq{Mi bakis kukon kaj ĝi bongustis.}\\
\engq{I baked a cake and it was tasty.}
\xe

\subsection{100--200 AMC}


\subsubsection{When glides occur as the lone onset of a syllable, they become fricatives}

Based on the maximum onset principle, this generally only occurs intervocalically or word-initially.

\begin{center}
    j w $\to$ \paljfric{} v / \$\_V
\end{center}

e.g., \espq{ejakuli} \bripa{\scstr e.ja\prstr ku.li} $\to$ \bripa{\scstr e.\paljfric a\prstr ku.li} but \espq{ajna} \bripa{\prstr aj.na} $\to$ \bripa{\prstr ej.na} but \espq{anta\u{u}a} \bripa{an\prstr to.wa} $\to$ \bripa{an\prstr to.va}

\subsubsection{Glides disappear after vowels but provide compensatory lengthening}

\begin{center}
    V$^{+front}$j $\to$ V\lgth\\
    V$^{+round}$w $\to$ V\lgth
\end{center}

Note that this change \emph{only} occurs when the glide part of the same syllable as the vowel in question---if the glide is instead the lone onset of the following syllable, it is affected by the previous sound change instead.

e.g., \espq{kajto} \bripa{\prstr kej.to} $\to$ \bripa{\prstr ke\lgth.to}, \espq{tuj} \bripa{tyj} $\to$ \bripa{ty\lgth}, \espq{vojmontrilo} \bripa{\scstr vøj.mon\prstr tri.lo} $\to$ \bripa{\scstr vø\lgth.mon\prstr tri.lo}, \espq{ambaŭ} \bripa{\prstr am.bow} $\to$ \bripa{am.bo\lgth} , \espq{eŭropa} \bripa{øw\prstr ro.pa} $\to$ \bripa{ø\lgth\prstr ro.pa}

\subsubsection{Glottal stop and glottal fricative merge}

\begin{center}
    \phipa{\glotstop} $\to$ \phipa{h}
\end{center}

e.g., \espq{heroo} \bripa{he\prstr\alvrap o.\glotstop o} $\to$ \bripa{he\prstr\alvrap o.ho}

\subsubsection{{\sc Lowlands:} Standalone obstruents are voiced intervocalically} 

\begin{center}
    p t c k t\tiebar s t\tiebar\esh{} f s \esh{} h $\to$ b d \paljstop{} g d\tiebar z d\tiebar\ezh{} v z \ezh{} \voih{} / V\_V
\end{center}

e.g., \espq{\^{s}ipo} \bripa{\prstr\esh i.po} $\to$ \bripa{\prstr\esh i.bo} 

\subsubsection{Intervocalic geminates become single occurrences of the consonant in question}

\begin{center}
    C$^1$C$^2$ $\to$ C$^1$ / C$^1$ = C$^2$
\end{center}

e.g., \espq{dis\^{s}uti} \bripa{di\esh\prstr\esh u.di} $\to$ \bripa{di\prstr\esh u.di}

\subsubsection{{\sc Lowlands:} Non-glottal fricatives are affricativized after nasal consonants}

\begin{center}
    f s \esh{} v z \ezh{} $\to$ p\tiebar f t\tiebar s t\tiebar\esh{} b\tiebar v d\tiebar z d\tiebar\ezh{} / N\_ 
\end{center}

e.g., \espq{komforti} \bripa{kom\prstr fo\alvrap.ti} $\to$ \bripa{kom\prstr p\tiebar fo\alvrap.ti}, \espq{bronza} \bripa{b\alvrap\vless on.za} $\to$ \bripa{b\alvrap\vless on.d\tiebar za}

\subsubsection{Vowels are nasalized before nasal consonants}

\begin{center}
    V $\to$ \~{V} / \_N
\end{center}

e.g., \espq{anka\u{u}} \bripa{\prstr\ahoh n.ko\lgth} $\to$ \bripa{\prstr\~{\ahoh}n.ko\lgth}

\subsection{200--300 AMC}

\subsubsection{{\sc Lowlands:} Nasal consonants are deleted when they precede obstruents}

\subsubsection{Ablaut}

If the last vowel in a word is rounded, the preceding vowel is rounded (regardless of intervening consonants).\\
e.g., \espq{kato} \bripa{\prstr ka.to} $\to$ \bripa{\prstr k\ahoh.to}, \espq{iros} \bripa{\prstr i.ros} $\to$ \bripa{\prstr y.ros}, \espq{ekzemplo} \bripa{ek\prstr sẽm.plo} $\to$ \bripa{ek\prstr sø̃m.plo}

If the last vowel in a word is front, the preceding vowel is fronted (regardless of intervening consonants).\\
e.g., \espq{havis} \bripa{\prstr ha.vis} $\to$ \bripa{\prstr h\aesh.vis}, \espq{ofte} \bripa{\prstr of.te} $\to$ \bripa{\prstr øf.te}, \espq{seksumi} \bripa{sek\prstr sũ.mi} $\to$ \bripa{sek\prstr sỹ.mi}

\begin{center}
    V$^{-front}$ $\to$ V$^{+front}$ / \_(C(C))\$(C(C(C)))V$^{+front}$\#\\
    V$^{-round}$ $\to$ V$^{+round}$ / \_(C(C))\$(C(C(C)))V$^{+round}$\#
\end{center}

\begin{center}
    \phipa{i}$^{+round} = $ \bripa{y}\\
    \phipa{e}$^{+round} = $ \bripa{ø}\\
    \phipa{a}$^{+round} = $ \bripa{\ahoh}\\
    \phipa{a}$^{+front} = $ \bripa{\aesh}\\
    \phipa{o}$^{+front} = $ \bripa{ø}\\
    \phipa{u}$^{+front} = $ \bripa{y}\\
\end{center}

\subsubsection{{\sc Highlands:} Word-initial syllables consisting of an obstruent or nasal, a vowel, and then a continuant in the syllable coda undergo metathesis of the schwa and continuant}

\begin{center}
    C$_1$VC$_2$ $\to$ C$_1$C$_2$V / \#\_(C)\$\\
    C$_1$ = obstruent or nasal, C$_2$ = continuant
\end{center}

e.g. \espq{forkuri} \bripa{fo\alvrap\prstr ku.\alvrap i} $\to$ \bripa{fro\prstr ku.\alvrap i}


\subsubsection{}

\subsection{300--400 AMC}


\subsubsection{\phipa{\alvrap} disappears in the coda and the preceding vowel, if short, undergoes compensatory lengthening}

\begin{center}
    V\alvrap{} $\to$ V\lgth{} / \_(C)\$
\end{center}

e.g., \espq{forkuri} (Lowlands) \bripa{fo\alvrap\prstr ku.\alvrap i} $\to$ \bripa{fo\lgth\prstr ku.\alvrap i}

\subsubsection{l-vocalization occurs when l occurs in the coda of a syllable}

\begin{center}
    l $\to$ w / V\_(C)\$
\end{center}

e.g., \espq{bulbo} \bripa{\prstr bul.bo} $\to$ \bripa{\prstr buw.bo}

\subsubsection{{\sc Highlands:} Standalone obstruents are voiced intervocalically}

\begin{center}
    p t c k t\tiebar s t\tiebar\esh{} f s \esh{} $\to$ b d \paljstop{} g d\tiebar z d\tiebar\ezh{} v z \ezh{} / V\_V
\end{center}

e.g., \espq{\^{s}ipo} \bripa{\prstr\esh y.po} $\to$ \bripa{\prstr\esh y.bo} (identical in Lowlands), \espq{dis\^{s}uti} \bripa{di\prstr\esh y.ti} $\to$ \bripa{di\prstr\ezh y.di} (cf. Lowlands \bripa{di\prstr\esh u.di}), \espq{anka\u{u}} \bripa{\prstr\~{\ahoh}.ko\lgth} $\to$ \bripa{\prstr\~{\ahoh}.go\lgth} (cf. Lowlands \bripa{\prstr\~{\ahoh}.ko\lgth})

\subsubsection{Unstressed short vowels become schwa}

e.g., \espq{Esperanto} \bripa{\scstr e.spe\prstr r\~{\ahoh}.to} $\to$ \bripa{\scstr e.sp\schwa\prstr r\~{\ahoh}.t\schwa}, \espq{tajfuno} \bripa{te\lgth\prstr fu.no} $\to$ \bripa{te\lgth\prstr fu.n\schwa}

\subsubsection{{\sc Lowlands:} After a nasal vowel, voiced stops become nasal stops}

\begin{center}
    b d \paljstop{} g $\to$ m n \egna{} \engma{} / \~{V}\_
\end{center}

e.g., \espq{embrio} \bripa{\~{\schwa}\prstr b\alvrap y.\paljfric\schwa} $\to$ \bripa{\~{\schwa}\prstr m\alvrap y\paljfric\schwa}

\subsection{400--500 AMC}

\input{sound_change_stages/amc_400.tex}

\subsection{500--600 AMC}

\input{sound_change_stages/amc_500.tex}

\subsection{600--700 AMC}

\input{sound_change_stages/amc_600.tex}

\subsection{700--800 AMC}

\input{sound_change_stages/amc_700.tex}

\subsection{800--900 AMC}

\input{sound_change_stages/amc_800.tex}

\subsection{900--1000 AMC}

\input{sound_change_stages/amc_900.tex}

\section{Sparky's Transcription}

I'll probably make an in-world orthography later, but regardless of whether I do, I need a roughly phonetic or phonemic transcription which I can use to not have to constantly write in IPA. That'll go here once I actually make it lol.

\clearpage
\section{Pronunciation Example}

The following is the Zamenhofian Esperanto text of the Lord's Prayer. Given that the text is highly ritualized, it has not been affected by the grammatical and lexical changes to the language, but it remains pronounced differently in the different dialects due to the sound changes that have occurred.

\subsection{Zamenhofican Esperanto c. 0 AMC}

\begin{multicols*}{2}
    \textit{Patro nia, kiu estas en la ĉielo, \\
    sanktigata estu Via Nomo. \\
    Venu Via regno. \\
    Fariĝu Via volo, \\
    kiel en la ĉielo, tiel ankaŭ sur la tero.}

    \textit{Nian panon ĉiutagan donu al ni hodiaŭ. \\
    Kaj pardonu al ni niajn ŝuldojn, \\
    kiel ankaŭ ni pardonas al niaj ŝuldantoj. \\
    Kaj ne konduku nin en tenton, \\
    sed liberigu nin de la malbono.}

    \textit{Ĉar Via estas la regno 
    kaj la potenco \\
    kaj la gloro eterne.}

    \textit{Amen.}
    
    \columnbreak

    \prstr pa.t\alvrap o \prstr ni.a\\
    \prstr ki.u \prstr e.stas en la t\tiebar\esh i\prstr e.lo\\
    \scstr sank.ti\prstr ga.ta \prstr e.stu \prstr \labrox i.a \prstr no.mo\\
    \prstr\labrox e.nu \prstr\labrox i.a \prstr\alvrap e.gno\\
    fa\prstr\alvrap i.d\tiebar\ezh u \prstr\labrox i.a \prstr\labrox o.lo\\
    \prstr ki.el en la t\tiebar\esh i\prstr e.lo\\
    \prstr ti.el \prstr an.kaw sur la \prstr te.\alvrap o

    \prstr ni.an \prstr pa.non \scstr t\tiebar\esh i.u\prstr ta.gan\\
    \prstr do.nu al ni ho\prstr di.aw\\
    kaj pa\alvrap\prstr don.u al ni \prstr ni.ajn \prstr\esh ul.dojn\\
    \prstr ki.el \prstr an.kaw ni pa\alvrap\prstr do.nas\\
    al \prstr ni.aj \esh ul\prstr dan.toj\\
    kaj ne kon\prstr du.ku nin en \prstr ten.ton\\
    sed \scstr li.be\prstr\alvrap i.gu nin de la mal\prstr bo.no

    t\tiebar\esh a\alvrap{} \prstr\labrox i.a \prstr e.stas la \prstr \alvrap e.gno\\
    kaj la po\prstr ten.t\tiebar so\\
    kaj la \prstr glo.\alvrap o e\prstr te\alvrap.ne

    \prstr a.men

    \end{multicols*}

\subsection{c. 100 AMC}

\prstr pa.t\alvrap o \prstr ni.ja\\
\prstr cju e\prstr stas en la \prstr t\tiebar\esh je.lo\\
\scstr sa\engma k.ti\prstr ga.ta \prstr e.stu \prstr vi.ja \prstr no.mo\\
\prstr ve.nu \prstr vi.ja \prstr\alvrap e.gno\\
fa\prstr\alvrap i.d\tiebar\ezh u \prstr vi.ja \prstr vo.lo\\
cjel en la \prstr t\tiebar\esh je.lo\\
tjel \prstr a\engma.kow sur la \prstr te.\alvrap o

\prstr ni.jan \prstr pa.non \scstr t\tiebar\esh ju\prstr ta.gan\\
\prstr do.nu al ni ho\prstr di.jow\\
kej pa\alvrap\prstr don.u al ni \prstr ni.jejn \prstr\esh ul.døjn\\
cjel \prstr a\engma.kow ni pa\alvrap\prstr do.nas\\
al \prstr ni.jej \esh ul\prstr dan.tøj\\
kej ne kon\prstr du.ku nin en \prstr ten.ton\\
sed \scstr li.be\prstr\alvrap i.gu nin de la mal\prstr bo.no

t\tiebar\esh a\alvrap{} \prstr vi.ja e\prstr stas la \prstr \alvrap e.gno\\
kej la po\prstr ten.t\tiebar so\\
kej la \prstr glo.\alvrap o e\prstr te\alvrap.ne

\prstr a.men


\subsection{c. 200 AMC}

\begin{multicols*}{2}

{\sc Lowlands}

\prstr pa.t\alvrap o \prstr ni.\paljfric a\\
\prstr cju e\prstr stas ẽn la \prstr t\tiebar\esh je.lo\\
\scstr sã\engma k.ti\prstr ga.da \prstr e.stu \prstr vi.\paljfric a \prstr no.mo\\
\prstr vẽ.nu \prstr vi.\paljfric a \prstr\alvrap e.gno\\
fa\prstr\alvrap i.d\tiebar\ezh u \prstr vi.\paljfric a \prstr vo.lo\\
cjel ẽn la \prstr t\tiebar\esh je.lo\\
tjel \prstr ã\engma.ko\lgth{} sur la \prstr te.\alvrap o

\prstr ni.\paljfric ãn \prstr pã.nõn \scstr t\tiebar\esh ju\prstr ta.gãn\\
\prstr dõ.nu al ni ho\prstr di.\paljfric o\lgth{}\\
ke\lgth{} pa\alvrap\prstr dõn.u al ni \prstr ni.\paljfric ẽ\lgth{}n \prstr\esh ul.dø̃\lgth{}n\\
cjel \prstr ã\engma.ko\lgth{} ni pa\alvrap\prstr dõ.nas\\
al \prstr ni.\paljfric e\lgth{} \esh ul\prstr dãn.tø\lgth{}\\
ke\lgth{} ne kõn\prstr du.gu nĩn en \prstr tẽn.tõn\\
sed \scstr li.be\prstr\alvrap i.gu nĩn de la mal\prstr bõ.no

t\tiebar\esh a\alvrap{} \prstr vi.\paljfric a e\prstr stas la \prstr \alvrap e.gno\\
ke\lgth{} la po\prstr dẽn.d\tiebar zo\\
ke\lgth{} la \prstr glo.\alvrap o e\prstr te\alvrap.ne

\prstr ã.mẽn

\columnbreak

{\sc Highlands}

\prstr pa.t\alvrap o \prstr ni.\paljfric a\\
\prstr cju e\prstr stas ẽn la \prstr t\tiebar\esh je.lo\\
\scstr sã\engma k.ti\prstr ga.ta \prstr e.stu \prstr vi.\paljfric a \prstr no.mo\\
\prstr vẽ.nu \prstr vi.\paljfric a \prstr\alvrap e.gno\\
fa\prstr\alvrap i.d\tiebar\ezh u \prstr vi.\paljfric a \prstr vo.lo\\
cjel ẽn la \prstr t\tiebar\esh je.lo\\
tjel \prstr ã\engma.ko\lgth{} sur la \prstr te.\alvrap o

\prstr ni.\paljfric ãn \prstr pã.nõn \scstr t\tiebar\esh ju\prstr ta.gãn\\
\prstr dõ.nu al ni ho\prstr di.\paljfric o\lgth{}\\
ke\lgth{} pa\alvrap\prstr dõn.u al ni \prstr ni.\paljfric ẽ\lgth{}n \prstr\esh ul.dø̃\lgth{}n\\
cjel \prstr ã\engma.ko\lgth{} ni pa\alvrap\prstr dõ.nas\\
al \prstr ni.\paljfric e\lgth{} \esh ul\prstr dãn.tø\lgth{}\\
ke\lgth{} ne kõn\prstr du.ku nĩn ẽn \prstr tẽn.tõn\\
sed \scstr li.be\prstr\alvrap i.gu nĩn de la mal\prstr bõ.no

t\tiebar\esh a\alvrap{} \prstr vi.\paljfric a e\prstr stas la \prstr \alvrap e.gno\\
ke\lgth{} la po\prstr tẽn.t\tiebar so\\
ke\lgth{} la \prstr glo.\alvrap o e\prstr te\alvrap.ne

\prstr ã.mẽn

\end{multicols*}

\subsection{c. 300 AMC}

\subsection{c. 400 AMC}

\subsection{c. 500 AMC}

\subsection{c. 600 AMC}

\subsection{c. 700 AMC}

\subsection{c. 800 AMC}

\subsection{c. 900 AMC}

\subsection{c. 1000 AMC}

\end{document}