\subsection{Intensification suffixes}

In Zamenhofian Esperanto, the augmentative \espq{-eg-} and diminutive \espq{-et-} are used as degree modifiers for adjectives, with \espq{-eg-} serving as an intensifier and \espq{-et-} as a downtowner regardless of the semantics of the adjective in question. 

Here, however, this usage begins to change, with adjectives being intensified using affixes that are semantically suited to their meaning. This is most apparently different with \espq{malgranda}, meaning \engq{small}. In Zamenhofian Esperanto, \espq{malgrandeta} means \engq{a little small} while \espq{malgrandega} means \engq{tiny}, whereas now \espq{malgrandega} has fallen out of usage and \espq{malgrandeta} has taken over its meaning of \engq{tiny}. The thought process behind this is that the diminutive of \engq{small} should be \emph{smaller} than the bare form of small, not larger, and thus it can serve as an intensifier here.

In addition to this example with \espq{malgranda}, other affixes like \espq{-el-} and \espq{-aĉ-} are used with adjectives that are semantically suited to them (e.g., \espq{belela} has replaced \espq{belega} and \espq{naŭzaĉa} has replaced \espq{naŭzega}).

\subsection{No more adjective agreement}

Pretty much what it says on the tin: adjectives no longer agree with the nouns they modify in number and case. 

\ex
\espq{La bela virinoj manĝis la dolĉa kukojn.}\\
\engq{The beautiful women ate the sweet cakes.}
\xe

\subsection{Emphatic negation}

In addition to ordinary negation with \espq{ne}, negation begins to be frequently emphasized by inclusion of \espq{neniel} as well. This roughly corresponds to inclusion of the English negative polarity item \engq{at all}.

\ex
\espq{Mi ne volas manĝi tion neniel.}\\
\engq{I don't want to eat that at all.}
\xe

\subsection{Use of \espq{si} logophorically}

In addition to its use as a simple reflexive, \espq{si} is expanded to be used as a logophoric pronoun, also being used in subclauses to indicate that the referent has not changed.

\pex
\a
\espq{Johano diris, ke si foriru.}\\
\engq{John$_i$ said that he$_i$ has to leave.}
\a
\espq{Johano diris, ke li foriru.}\\
\engq{John$_i$ said that he$_j$ has to leave.}
\xe