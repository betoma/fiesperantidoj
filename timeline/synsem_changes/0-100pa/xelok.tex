
\subsubsection{Pronoun Usage}

In contexts where it's unclear whether the referent is male or female, using a gendered \espq{li} or \espq{ŝi} is now dispreferred. The current preference is to use \espq{tiu}. However, like singular-they in English, this isn't generally done with explicitly gendered referents, but rather only with those whose gender is unknown or unspecified.

\ex
\espq{Se via studento scias, kiu ĝin faris, tiu devas paroli.}\\
\engq{If your student$_i$ knows who did it, they$_i$ should talk.}
\xe

Note that, where appropriate, \espq{si} and its forms are still used.

\ex 
\espq{Iu preterlasis sian pluvombrelon!}\\
\engq{Someone$_i$ forgot their$_i$ umbrella!}
\xe

\subsubsection{Conditional Participles}

The formerly-unofficial conditional participle \textit{-unta} is now common and unremarkable. They are not generally used predicatively, as using the conditional verb form with other participles serves those purposes fine, but they are widely used attributively and nominalized.

\pex
\a
\espq{La mortunta knabino}\\
\engq{The girl who would/could have died}
\a
\espq{La regunto}\\
\engq{The would-be ruler}
\xe

\subsubsection{Verbalization of predicate adjectives}

Rather than using \textit{esti} as a copula, adjectives are now generally used directly as stative verbs. Nominals continue to use \textit{esti}, however, and \textit{esti} can be included for emphasis for predicate adjectives.

\ex
\espq{La birdo estas blua} $\to$ \espq{La birdo bluas.}\\
\engq{The bird is blue.}
\xe

\subsubsection{Widespread adoption of `far'}

To avoid the ambiguity of the Zamenhofian Esperanto \espq{de}, \espq{far} is adopted as a shortening of \espq{fare de}, to indicate that something was done/made by someone rather than merely owned by or associated with them.

\pex
\a
\espq{La bindaĵo de la libro de Maria ruĝas.}\\
\engq{The cover of Maria's book (a book which Maria owns but didn't necessarily write) is red.}
\a
\espq{La bindaĵo de la libro far Maria ruĝas.}\\
\engq{The cover of Maria's book (a book which Maria wrote) is red.}
\xe

\subsubsection{Widespread adoption of `cit'}

To attribute a quote to someone, the preposition \espq{cit} is used (again, replacing Esperanto \espq{de}).

\ex
\espq{\engq{Rompu, rompu la murojn inter la popoloj!} cit Zamenhof estas inspiranta citaĵo.}\\
\engq{`Break, break the walls between the peoples!' by Zamenhof is an inspiring quotation.}
\xe

\subsubsection{Country names don't end in \textit{-ujo}}

All country names ending in \textit{-ujo} are replaced with alternatives ending in \textit{-io}. 
\ex
\espq{Francujo} $\to$ \espq{Francio}\\
\engq{France}
\xe
Unrelatedly, a few country names change entirely due to reforming their names being a thing in 2010s Esperanto already:
\ex
\espq{Finnlando} $\to$ \espq{Suomio}\\
\engq{Finland}
\xe

\subsubsection{Free variation between presence and absence of linking -o- in compounds}

Linking -o- isn't completely lost, but it's beginning to be less common than it is in ordinary Esperanto. Forms of the same word with and without linking -o- are common and generally occur in free variation.

\ex
\espq{rozokoloro} \til{} \espq{rozkoloro}
\xe

\subsubsection{Identical adjectival and nominal forms dispreferred}

Words where the adjectival and nominal forms are identical and only distinguished by the difference between the final \textit{-o} and \textit{-a} are highly dispreferred---as are affixes being applied with both an \textit{-o} or \textit{-a}. The less `core' part-of-speech for a given root is formed using another affix that suits it semantically. Popular options include \espq{-ulo} `person', \espq{-ano} `member of', \espq{-eco} `quality of, -ness', \espq{-ema} `inclined toward', etc. Sometimes compounding performs the same role---for instance, \espq{roza} \engq{pink (adj.)} vs. \espq{roz(o)koloro} \engq{pink (n.)} vs. \espq{roz(o)floro} \engq{rose}

\subsubsection{Less-recognized/Ido-loaned suffixes more common}

\paragraph{\espq{-el-}}

X\textit{-elo} = \engq{good/beautiful X}: \espq{skribo} \engq{writing}, \espq{skribelo} \engq{calligraphy}

\paragraph{\espq{-oz-}} 

X\textit{-oza} = \engq{full of X}: \espq{monto} \engq{mountain}, \espq{montoza} \engq{mountainous}

\paragraph{\espq{-end-}}

X\textit{-enda} = \engq{needing to be X-ed}: \espq{pagi} \engq{to pay}, \espq{pagenda} \engq{needing paid}

\subsubsection{\espq{ali-} is an official correlative now}

The forms \espq{alio}, \espq{aliu}, \espq{alia}, \espq{aliam}, \espq{aliom} \espq{aliel}, etc. are now officially sanctioned rather than not being official. Because \espq{alie} already exists, the form for `another place' is \espq{aliloke}.

\subsubsection{Pro-drop-ness begins: if \espq{ĝi} would be the subject, don't bother}

The inanimate 3rd person pronoun \espq{ĝi} is now pretty much universally dropped when it would be the subject of a sentence. I'm not actually sure to what extent this is done in vanilla Esperanto tbh---I know it's already a think for zero-valency verbs like weather verbs, but now it's a thing for any sentence where \espq{ĝi} would be the subject.

\ex
\espq{Mi bakis kukon. Bongustis.}\\
\engq{I baked a cake. It was tasty.}
\xe

This is \emph{not} true in instances where it would be blocked by a subject-pivot, though!

\pex
\a
\ljudge{\#}
\espq{Mi bakis kukon kaj bongustis.}\\
\engq{I baked a cake and was tasty.}
\a
\espq{Mi bakis kukon kaj ĝi bongustis.}\\
\engq{I baked a cake and it was tasty.}
\xe