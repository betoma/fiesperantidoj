
\subsection{Gendered third-person animate pronouns replaced with demonsrative}

Zamenhofian Esperanto uses the gendered pronouns \textit{li} `he' and \textit{ŝi} `she' for 3rd person animate referents. In addition, it uses the reflexive pronoun \textit{si} `oneself' to make it clear when the subject acts upon something within the same clause. For example,
\begin{quote}
    \textit{Johano$_i$ kaj Marko$_j$ batalis, ĉar li$_i$ prenis lian$_j$ forkon.}\\
    John and Mark fought because he (John) grabbed his (Mark's) fork.\\
    \\
    \textit{Johano$_i$ kaj Marko$_j$ batalis, ĉar li$_i$ prenis sian$_i$ forkon.}\\
    John and Mark fought because he (John) grabbed his (own) fork.
\end{quote}
However, despite the presence of the reflexive pronoun, its confinement to within the same clause makes it unclear who is being references in some clauses:
\begin{quote}
    \textit{Johano$_i$ kaj Marko$_j$ rendevuis, kaj li$_i$ diris, ke li$_\{i,j\}$ forigu.}\\
    John and Mark met up and he (John) said that he (which one?) had to leave.
\end{quote}

Both \textit{li} and \textit{ŝi} are replaced with the genderless \textit{tiu}, a correlative which means `that one (person)'. Though this could increase the ambiguity in pronoun reference described above by leveling the gender distinction, it's made up for by more extensive use of \textit{si} as a logophoric pronouun (rather than a mere reflexive).
\begin{quote}
    \textit{Johano$_i$ kaj Marko$_j$ rendevuis, kaj si$_i$ diris, ke tiu$_j$ foriru.}\\
    John and Mark met up and he(John) said that he (Mark) had to leave\\
    \\
    \textit{Johano$_i$ kaj Marko$_j$ rendevuis, kaj si$_i$ diris, ke si$_i$ foriru.}\\
    John and Mark met up and he(John) said that he (John) had to leave
\end{quote}

\subsection{Conditional participles are now used to combine conditionality with tense}

Conditional participles formed with \textit{-unta}, never officially allowed in Esperanto proper, are now perfectly fine. They're generally not used predicatively in present tense, but they are preferred for the past and future tenses. This allows the granular expression of sentiments that were somewhat difficult before.
\begin{quote}
    \textit{Mi irus hejmen, sed mi iris al bordelo.}\\
    I would go home, but I went to the brothel.\\
    \\
    \textit{Mi estis irunta hejmen, sed mi iris al bordelo.}\\
    I would have gone home, but I went to the brothel.
\end{quote}

\textit{estis X-unta} largely replaces \textit{estus X-inta} (and the same is the case for the equivalent future tense). However, this is not the case when a modal is used as an auxiliary rather than \textit{esti}---for instance, \textit{povus X-inta} is still used for `could have X-ed'.

At the same time, the future tense form \textit{X-os} is replaced in many contexts with the present tense plus the prospective aspect \textit{estas X-onta}, in addition to the existing uses of the prospective. However, this only occurs for dynamic verbs and statives and modals continue to use the proper future tense.

\subsection{\textit{esti} now only serves as a nominal copula. Predicative adjectives are now universally conjugated as stative verbs.}

\begin{quote}
    \textit{La birdo estas blua} $\to$ \textit{La birdo bluas}
\end{quote}
This also applies to participles.
\begin{quote}
    \textit{La arbo estas falinta} $\to$ \textit{La arbo falintas}
\end{quote}

\subsection{Inherently gendered words are changed}

The formerly-default masculine readings of certain familial terms are now formed using the \textit{-iĉ-} affix to match how the feminine forms are formed with \textit{-in-}. Gender neutral forms are now formed with \textit{ge-} (which can now be prefixed to the singular to mean `of undetermined gender').
\begin{quote}
\textit{patro} $\to$ \textit{patriĉo}\\
\textit{patrino} $\to$ \textit{patrino}\\
\textit{unu el la gepatroj} $\to$ \textit{gepatro}
\end{quote}
Adjectival forms are not altered in this way---for instance, `brotherly love' is still \espq{frata amo}, not \espq{fratiĉa amo} or \espq{gefrata amo}.

This doesn't affect words where there are already existing lexically-distinct variants for both gender-neutral and feminine forms of the words. For instance, \textit{viro} `man' does not become \textit{viriĉo}, because the gender-neutral \textit{persono} `person' and the feminine \textit{ino} `woman' serve the roles of gender-neutral and feminine forms.

\subsection{Some new prepositions are innovated/no longer considered nonstandard}

\subsubsection{\textit{far} `by' replaces \espq{fare de}}

This is done to avoid the ambiguity caused by the Zamenhofian Esperanto \espq{de}.
\begin{quote}
    \textit{Kuko manĝota de Jena.}\\
    Jen's cake that is about to be eaten (by someone, not necessarily Jen).\\
    \\
    \textit{Kuko manĝota far Jena.}\\
    The cake that is about to be eaten by Jen.
\end{quote}
\espq{far} explicitly states that the action was performed by the objective of the preposition, in constrast with the vaguer \espq{de}.

\subsubsection{\textit{cit} `by'/`quot.'}

\espq{cit} is used to introduce the author/speaker of a quotation (again replacing \espq{de}).
\begin{quote}
    \textit{``Rompu, rompu la murojn inter la popoloj!'' cit Zamenhof estas inspiranta citaĵo.}\\
    ``Break, break the walls between the peoples!'' by Zamenhof is an inspiring quotation.
\end{quote}

\subsubsection{Introduction of \textit{na}}

\espq{na} is introduced as an accusative preposition to be used in contexts where an ordinary accusative is ambiguous; i.e., to mark the `lower' accusative in a double-accusative construction, or for names and other words that cannot take \textit{-n}.

There's an early difference between highlands and lowlands language here---highlands speakers studiously avoid double accusatives as well but use \textit{al} on the higher argument instead of applying \textit{na}. They do, however, still use \textit{na} for words that cannot take the accusative \textit{-n}.
\begin{quote}
    \textit{Mi instruas vin tion.} $\to$\\
    \textsc{Lowlands:} \textit{Mi instruos vin na tion.}\\
    \textsc{Highlands:} \textit{Mi lernigos al vi tion.}\\
    \\
    \textit{Mi eniras Walmart.} $\to$ \textit{Mi eniras na Walmart.}
\end{quote}

\subsection{Country names don't end in \textit{-ujo}}

All country names ending in \textit{-ujo} are replaced with alternatives ending in \textit{-io}. 
\begin{quote}
    \textit{Francujo} $\to$ \textit{Francio}
\end{quote}
Unrelatedly, a few country names change entirely due to reforming their names being a thing in 2010s Esperanto already:
\begin{quote}
    \textit{Finnlando} $\to$ \textit{Suomio}
\end{quote}

\subsection{Identical adjectival and nominal forms dispreferred}

Words where the adjectival and nominal forms are identical and only distinguished by the difference between the final \textit{-o} and \textit{-a} are highly dispreferred---as are affixes being applied to both an \textit{-o} or \textit{-a}. The less `core' part-of-speech for a given root is formed using another affix that suits it semantically. Popular options include \espq{-ulo} `person', \espq{-ano} `member of', \espq{-eco} `quality of, -ness', \espq{-ema} `inclined toward', etc. Sometimes compounding performs the same role.

\subsection{Adjectives no longer agree what they modify}

Adjectives are no longer marked for case or number, and their reference is determined solely through their placement next to what they modify.
\begin{quote}
    \textit{La belaj virinoj} $\to$ \textit{La bela inoj}
\end{quote}

\subsection{The usage of \textit{-iĝ-} broadens}

\espq{-iĝ-} already has a pretty broad meaning in Zamenhofian Esperanto, but its usage as a mediopassive rather than simply an anticausative is more widespread and generally accepted. Passive forms are more often replaced by these forms than in ordinary Esperanto.
\begin{quote}
    \textit{La kuko bakiĝis.}\\
    The cake bakes.\\
    \\
    \textit{La kuko estas bakita de mia patrino} $\to$\\
    \textit{La kuko bakiĝis far mia patrino}
\end{quote}

Deriving verbs meaning `to become X' from nouns and adjectives using \textit{-iĝi} remains popular. Periphrastic causative/inchoatives can still be formed with \espq{igi} and \espq{fariĝi}, but they're done less often than in Esperanto and really only occur when adding the affix is difficult, impossible, or ambiguous.

\subsection{Less-recognized/Ido-loaned suffixes more common}

\subsubsection{Laudatory \espq{-el-}}

\textit{skribo} `writing', \textit{skribelo} `calligraphy'

\subsubsection{\textit{-oz-} `full of'}

\textit{monto} `mountain', \textit{montoza} `mountainous'

\subsubsection{\textit{-end-} `needing done'}

\textit{pagi} `to pay', \textit{pagenda} `needing paid'

\subsection{\espq{ali-} is an official correlative now}

The forms \espq{alio}, \espq{aliu}, \espq{alia}, \espq{aliam}, \espq{aliom} \espq{aliel}, etc. are now officially sanctioned rather than not being official. Because \espq{alie} already exists, the form for `another place' is \espq{aliloke}. However, the same thing occurs with other correlatives: \espq{tiuloke}, \espq{ĉiuloke}, \espq{kiuloke}, etc.

\subsection{Polar questions formed by subordination}

Polar questions are now generally formed by subordinating the statement under \espq{Ĉu veras, ke...?} (`Is it true that...?')

\begin{quote}
\espq{Ĉu vi deziras vaflon?} $\to$\\
\espq{Ĉu veras, ke vi deziras vaflon?}\\
Do you want a waffle?
\end{quote}

\subsection{Left dislocation af}

Left dislocation is now way more common than in vanilla Esperanto. The dislocated element precedes \emph{everything} in a given sentence, including sentence-initial particles like \espq{ĉu}. As in English, left dislocation marks a topic, and it is generally limited to items that are already given in or inferrable from the discourse---as a result, it's also particularly often used for generics

\begin{quote}
    \textit{Keksoj, ili enhavas tro multe da sukero}\\
    Cookies, they contain too much sugar.\\
    \\
    \textit{La aŭto, ĉu vi lavis ĝin?}\\
    The car, did you wash it?
\end{quote}