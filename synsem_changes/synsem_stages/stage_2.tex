
\subsection{Number}

Grammatical number is not completely done away with, but it becomes far more marked than it ever was in Esperanto proper. Rather than simply being used for any plural quantity of a noun, plurals tend to mark generics in particular, with grammatical number marking being dropped for any usage that refers to a plural group of specific referants rather than to the class as a whole.
\begin{quote}
    \textit{Najbara katoj pli afablas ol sovaĝa katoj.}\\
    Neighborhood cats (in general) are more friendly than wild cats.\\
    \\
    \textit{L'najbara kato afablas.}\\
    The neighborhood cat(s) are friendly.
\end{quote}

Specific quantities of more than one of something are marked with either the relevant number or something quantificiational like \espq{iom da}, \espq{multe da}, \espq{ĉiom da}, etc.
\begin{quote}
    \textit{Mi manĝigis (iom da) najbara kato.}\\
    I fed (some) neighborhood cat(s).
\end{quote}

\subsection{Definiteness \& Case}

(Note: might change the object incorporation later to better fit with how noun incorporation actually works.)

In Lowlands speech, \textit{la} cliticizes onto the front of the noun phrase it would precede as \textit{l'} (pronounced with a linking schwa for consonant-initial words). As with \espq{la} in vanilla Esperanto, it is still used when nouns are omitted.
\begin{quote}
    \textit{Mi deziras l'ruĝa}\\
    I want the red one(s).
\end{quote}
Because sound changes result in the loss of accusative endings across the board, \textit{na} (now \textit{n'})'s use expands in Lowlands speech, taking over all accusatives.
\begin{quote}
    \textit{Mi havas ruĝan fiŝon.} $\to$ \textit{Mi havas n'ruĝa fiŝo.}
\end{quote}
If the object in question is definite, \textit{l'} and \textit{n'} merge into \textit{nal}.
\begin{quote}
    \textit{Mi havas la fiŝon.} $\to$ \textit{Mi havas nal fiŝo}
\end{quote}
This type of combination occurs with most occurrences of a definite article following a vowel-final preposition.

In Highlands speech, the accusative and definite are both lost altogether.
\begin{quote}
    \textit{Mi havas (la) fiŝon} $\to$ \textit{Mi havas fiŝo.}
\end{quote}

In both varieties, an object without any definiteness marking or adjectives will instead by incorporated onto the verb, like so:
\begin{quote}
    \textit{Mi havas fiŝon.} $\to$ \textit{Mi fiŝhavas.}
\end{quote}
Due to the left-dislocation constraints, generally an object that is given or inferred (and thus probably definite) will occur sentence-initially, whereas a newly introduced object will either be attached to the verb or will occur after the verb.
\begin{quote}
    \textit{Mi havas la ruĝan fiŝon} $\to$
    \textit{Ruĝa fiŝo mi havas (nal) ĝi.}\\
    \\
    \textit{Mi havas ruĝan fiŝon} $\to$
    \textit{Mi havas (n')ruĝa fiŝo}\\
\end{quote}

\subsubsection{Pronoun}

something obviative or switch reference with \espq{tiu}, \espq{si}, and \espq{aliu} but dear god I need to re-figure this out

introduce new plural and inclusive pronouns from \espq{ni ĉiuj} and \espq{vi ĉiuj}

Pronominal direct objects are pretty much always cliticized onto the front of the verb, syntactically serving as syntactic words preceding the VP, but phonologically attaching more like affixes (and thus phonologically affecting the word in question).
\begin{quote}
    \textit{Mia amiko kisis min} $\to$ \textit{Mia amiko minkisis}
\end{quote}

Maybe the language is now pro-drop? Haven't decided yet.

\subsection{Possession}

\espq{de} is no longer productively used for possession---it has been replaced by cliticizing the possessive pronoun that agrees with the possessor onto the possessed NP. This essentially turns this into a head-marking rather than dependent-marking construction. In the 3rd person, when the possessor generally is specified, they're simply placed before the possessee.
\begin{quote}
    \textit{manĝaĵo de kato} $\to$ \textit{kato ĝia=manĝaĵo}
\end{quote}
Definiteness and case-marking clitics attach to the phrase after this operation occurs, meaning they precede the possession clitic and the possessor noun if present.
\begin{quote}
    \textit{Mi deziras la pilko de la hundo.} $\to$ \textit{L'hundo ĝia=pilko}
\end{quote}
(I'll figure out which pronoun animate referent's use later.)

\subsection{Demonstratives}

Demonstratives are no longer made using a form based on \espq{tiu} but are instead marked by following the noun with \espq{tie}.

\subsection{(Medio-)Passives}

The passive verb forms are no longer used at all and have been pretty much wholly replaced with the mediopassive \espq{-iĝ-}

\subsection{Evolution of Esperanto aspects}

do more with this later anyway

perfect -> completive/resultative?
only used with telic verbs except when it's sometimes used with atelic predicates, in which case it bestows implications of culmination/accomplishment, adding implications of a fixed amount of the activity that needed to be (and was) completed. Cannot be used for statives tho.

progressive -> delimiative?
can only be used for atelic non-stative predicates
presents a "holistic transitory situation", doing something "a little bit" or "for a sec". Also can carry connotations of tentativeness.

Former habitual marker now generalized to an inflectional habitual aspect. 
Does it retain its old use as a durative `on and on'?

Nominalizations of verbs are derived from the former \espq{-ado} with greater frequency than in Esperanto proper.

\subsection{Negation}

Jespersen's cycle---\espq{ne} has been replaced with \espq{neniel}

\subsection{Adjectives}

Single attributive adjectives are more often than not fused onto the fronts of nouns as pseudo-compounds. If they cannot form something sensible this way (if there's more than one adjective or if you want to draw attention to the adjective in some way), they now generally follow the noun.

\subsubsection{Comparative Form}

add affixes---figure out deets later

\subsubsection{Degree Modifiers}

bare comparative
\espq{plej-}
\espq{tre-}
\espq{plen-}

\subsection{Modal Preverbal Clitics}

grab these from doc later
